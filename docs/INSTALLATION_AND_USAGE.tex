% Installation and Usage - LaTeX version
\documentclass[11pt,a4paper]{article}
\usepackage[utf8]{inputenc}
\usepackage[T1]{fontenc}
\usepackage{lmodern}
\usepackage[a4paper,margin=1in]{geometry}
\usepackage{hyperref}
\usepackage{fancyhdr}
\usepackage{enumitem}
\usepackage{titlesec}
\usepackage{caption}
\usepackage{bera} % monospace font
\titleformat{\section}{\Large\bfseries}{\thesection}{1em}{}
\title{Installation and Usage}
\author{Ordonnanceur-multi-tache-de-processus}
\date{\today}
\pagestyle{fancy}
\fancyhf{}
% ensure enough header space for fancyhdr
\setlength{\headheight}{14pt}
\lhead{Ordonnanceur-multi-tache-de-processus}
\rhead{Installation and Usage}
\cfoot{\thepage}
\begin{document}
\maketitle
\tableofcontents
\vspace{1em}

\section{Overview}
This document explains how to install, build, and run the Multi-Process Scheduler Simulator.

The simulator visualizes process scheduling algorithms (TUI and GUI modes). Scheduling algorithms are implemented as separate modules and dynamically loaded at runtime.

\subsection*{Repository layout (key files)}
\begin{itemize}[left=0pt]
  \item \texttt{src/main.c} --- application entry point, CLI and menu handling
  \item \texttt{src/processus.h} --- \texttt{Processus} struct and shared definitions
  \item \texttt{src/affichage.c} --- Gantt chart and display logic
  \item \texttt{src/politiques/} --- scheduling algorithm implementations
  \item \texttt{src/processus.txt} --- example process input file
  \item \texttt{build/} --- compiled artifacts created by \texttt{make}
\end{itemize}

\section{Prerequisites}
Recommended platform: Linux / WSL (Windows Subsystem for Linux).

Tools required:
\begin{itemize}
  \item \texttt{gcc} (C compiler)
  \item \texttt{make}
\end{itemize}

\section{Build (compile) the project}
\begin{enumerate}
  \item Open a terminal in the \texttt{src/} directory:
  \begin{verbatim}
cd path/to/Ordonnanceur-multi-tache-de-processus/src
  \end{verbatim}
  \item Run \texttt{make}:
  \begin{verbatim}
make
  \end{verbatim}
This compiles the main program and the scheduling modules. The compiled shared libraries are placed into \texttt{build/politiques/} and the executable \texttt{main} is produced.
\end{enumerate}

\section{Process input file format}
Each process is one line with the following fields separated by spaces:
\begin{verbatim}
process-name arrival-time execution-time initial-priority
\end{verbatim}

Example (\texttt{processus.txt}):
\begin{verbatim}
P1 0 5 1
P2 1 3 2
P3 2 4 1
\end{verbatim}

\section{Run the simulator}
There are two main modes:

\subsection{TUI (Terminal UI) mode}
Show a terminal-based menu and results in the terminal. Provide the process file as argument:
\begin{verbatim}
./main --tui processus.txt
\end{verbatim}

\subsection{GUI mode}
Open the graphical interface:
\begin{verbatim}
./main --gui processus.txt
\end{verbatim}

\section{Add a new scheduling algorithm}
\begin{enumerate}
\item Create a \texttt{.c} file in \texttt{src/politiques/}, e.g. \texttt{my\_algorithm.c}.
\item Include the shared header and implement \texttt{ordonnancer}:
\begin{verbatim}
#include "../processus.h"

void ordonnancer(Processus T[], int n) {
    // scheduling logic here
}
\end{verbatim}
\item Run \texttt{make} again; the makefile auto-detects new files and compiles them to \texttt{build/politiques/}.
\end{enumerate}

\section{Troubleshooting}
\begin{itemize}
  \item \texttt{make: command not found} --- install build tools with your package manager.
  \item \texttt{Permission denied} when running \texttt{./main} --- run \texttt{chmod +x main} or run via WSL/appropriate shell.
  \item \texttt{process file not found} --- ensure path is correct; use absolute path if needed.
  \item Compilation errors in \texttt{politiques/} --- confirm each module implements \texttt{void ordonnancer(Processus T[], int n)} and includes \texttt{../processus.h}.
\end{itemize}

\section{Contact / Contributing}
Feel free to open issues or PRs for adding algorithm implementations, improving visuals or GUI, or automating PDF generation via CI.

\vfill

\end{document}
